% \vspace{-4.0\baselineskip}
% \vspace{4 cm}
% \begin{flushright}
%    {\LARGE \textbf{ABSTRACT}}
% \end{flushright}

% \vspace{-0.5\baselineskip}
% \noindent\rule{\linewidth}{2pt}
\newgeometry{tmargin=2.5cm, lmargin=38mm, rmargin=25mm, bmargin=25mm}
\thispagestyle{empty}
%\vspace*{0.1cm}
\textbf{\begin{flushright}
		{\LARGE \textbf{Abstract}}
\end{flushright}}
\vspace{-0.8\baselineskip}
\noindent\rule{\linewidth}{2pt}

\par
In education, Exam paper evaluation is a tedious processs for the teachers and professors. 
It is a time consuming process and also prone to human errors. On the other hand the 
evaluation process is crucial to students as it somewhat affects their near future. As this is a manual process,
this process is depends on several parameters, like the teacher's mood, teacher's idea about the student,
the student's handwriting etc. Although these create various biases in the evaluation process, they are 
inevitable, these kinds of human error will remain till humans grade the papers. In this project, we 
created a model to automate the evaluation process using machine learning techniques and Natural Language Processing.
We have used the model to predict the results of some exam papers and got comparable result with the actual results.
Moreover this process grants several more benifits rather than human evaluation.


% %%%%%%%%%%%%%%%%%%%%%%%%%%%%%%%%%%%%%%%%%%%%%%%%%%%%%%%%%%%%%%%%%%%%%%%%%%%%%%%%%%%%%%%%%%%%%%%%%%%%%%%%%%%%%%%%%%%%%%%%%%%%%%%%%%%%%%%%%%%%%%%%%%%%%%%%%%%%%%%%%%%%%%%%%%%%%%%%%%%%%%%%%%%%%%%%%%%%%%%%%%%%%%%%%%%%%%%%%%%%%%%%%%%%%%%%%%%%%%%%%%%%%%%%%%%%%%%%%%%%%%%%%%%%%%%%%%%%%%%%%%%%%%%%%%%%%%%%%%%%%%%%%%%%%%%%%%%%%%%%%%%%%%%%

% \newpage
% %\vspace{1.26 cm}
% %\vspace{5cm}
%  %\vspace*{0.1cm}  % Forces space at the very top of the page
% \newgeometry{tmargin=3.3cm, lmargin=38mm, rmargin=25mm, bmargin=25mm}
% % \begin{hindi}
% % \begin{flushright}
% %     {\LARGE \textbf{सार}}
% % \end{flushright}
% % \end{hindi}
% \vspace{-1\baselineskip}  % Adjust this to reduce or increase spacing after the title
% \noindent\rule{\linewidth}{2pt}

% % \begin{hindi}
% % इस पृष्ठ पर अपना अवधारणा हिंदी में लिखना है।
% % \end{hindi}