%%%%%%%%%%%%%%%%%%%%%%% CHAPTER - 1 %%%%%%%%%%%%%%%%%%%%\\
\chapter{Literature Review}
\label{C2} %%%%%%%%%%%%%%%%%%%%%%%%%%%%

%\noindent\rule{\linewidth}{2pt}
%%%%%%%%%%%%%%%%%%%%%%%%%%%%%%%%%%%%%%%%%%%%%%%%%%%%%%%%%%%%%%%%%%%%%%%%%%%%%%%%%%%%%%%%%%%%%%%%%%%%%%%%%%%%%%%%%%%%%%%%%%%%%%%%%%%%
%%%%%%%%%%%%%%%%%%%%%%% CHAPTER - 1 %%%%%%%%%%%%%%%%%%%%\\
%%%%%%%%%%%%%%%%%%%%%%%%%%%%%%%%%%%%%%%%%%%%%%%%%%%%%%%%%%%%%%%%%%%%%%%%%%%%%%%%%%%%%%%%%%%%%%%%%%%%%%%%%%%%%%%%%%%%%%%%%%%%%%%%%%%%


% \section{Section}
% \par 
% \subsection{Subsec}

% \par 
% \section{Table}
% \begin{table}[H]
%     \caption{Table}
%     \centering
%     \begin{tabular}{|c|>{\centering\arraybackslash}p{0.7\linewidth}|}
%     \hline
%     \textbf{Methods} & \textbf{Limitations} \\ % Bold headings
%     \hline
%     Method 1 & 
%     \begin{itemize}
%         \item More time-consuming than other methods.
%         \item Results suffer from subjective judgments of the inspector.
%     \end{itemize} \\
%     \hline
%     Method 2 & 
%     \begin{itemize}
%         \item Sensitive to the shape and size of the structure.
%         \item Needs highly careful attention during the test.
%         \item Limited to testing distance and the number of surfaces.
%     \end{itemize} \\
%     \hline
%    Method 3 & 
%     \begin{itemize}
%         \item Impossible to test on structures that are out of the scanner's line of sight.
%         \item Implementation cost is high.
%         \item Sensitive to the environment for setting up of equipment.
%     \end{itemize} \\
%     \hline
%     Method 4 & 
%     \begin{itemize}
%         \item Requires certain safety parameters due to hazardous ionising radiation.
%         \item Two-sided access to the structure is needed.
%         \item Relatively expensive testing equipment.
%     \end{itemize} \\
%     \hline
%     Method 5 & 
%     \begin{itemize}
%         \item Sensitive to environment noises and illuminated conditions.
%     \end{itemize} \\
%     \hline
%     \end{tabular}
%     \label{tab1}
% \end{table}

\vspace{-0.75cm}
\par
Several different types of word/sentence embedding were studied to understand the context of the text. 
And then creating a system to evaluate answer scripts accordingly. One of the most famous
embeddings is BERT \cite{bert}, they used transformers to train the model over a huge corpus to generate
word embeddings and sentence embeddings. In this model, each word/sentence is represented by a vector of 
dimension 768. This is the base model. There is another model called SciBERT \cite{scibert},
which is trained on a huge corpus of scientific writings on \href{https://www.semanticscholar.org}{semanticscholar.org}.
This works better on scientific text.
\par
There is another approach to word embeddings named Word2Vec \cite{word2vec}. This creates word embeddigs 
with a dimension specified by the user. This has been extended to create a model named Doc2Vec \cite{doc2vec} which can generate 
sentence embeddings from a corpus of data. This has been the main tool in our project. Research on Paragraph Vector
has also been done by \cite{p2} which is worth mentioning, they explained use of paragraph vectors in
Natural Language Processing.
\par
Research on automated exam paper evaluation is also done previously by \cite{p1} by detecting and counting 
keywords and calculating similarity using Jaccard similarity and WMD. 